\documentclass{amsart}
\renewcommand{\familydefault}{\sfdefault}
\usepackage{helvet}
\title{Angel English}
%\author{Angel English}
\address{16 Cuming st Yarraville 3013}
\email{\href{mailto:angelenglish@gmail.com}{angelenglish@gmail.com}, \href{mailto:1wsx10@gmail.com}{1wsx10@gmail.com}}
%\date{August 20, 2019}

% Check if we are compiling under latex or pdflatex, and include the
% appropriate graphics package
\ifx\pdftexversion\undefined
  \usepackage[dvips]{graphicx}
\else
  \usepackage[pdftex]{graphicx}
\fi

\usepackage{hyperref}

% Define the theorem environments
\newtheorem{theorem}{Theorem}[section] % numbered like the section
\newtheorem{lemma}[theorem]{Lemma} % numbered like the theorems
\newtheorem{proposition}[theorem]{Proposition}
\newtheorem{corollary}[theorem]{Corollary}
\theoremstyle{definition} % styled differently... not italicized
\newtheorem{definition}[theorem]{Definition}
\newtheorem{conjecture}[theorem]{Conjecture}

% Let \sign show roman-style characters in math mode
\newcommand{\sign}{\mathop{\mathrm{sign}}}

%\raggedbottom % Makes the bottom margin more flexible (helpful for pictures)

\begin{document}
\maketitle

Computer Science graduate specializing in C++ programming.
I'm at home using linux at the command line, so I'm not afraid to get
my hands dirty. I'm particular about implementing things the right
way so that they fit with future systems, rather than needing to be
overhauled. I like using C++ because it lets you get into the
performance detailes when needed, but also gives abstractions for
use most of the time.

\section{Qualifications}

\begin{itemize}
	\item
		RMIT
		\textsubscript{
			\href{https://www.myequals.net/sharelink/9ea57606-24b8-45e4-8030-4db56b0bee20/1c373231-4325-435c-b587-dab0f3481f64}{Transcript}
		}
		\begin{itemize}
			\item
				\href{https://www.myequals.net/sharelink/772babfa-f554-43b1-af4c-294eecf0b1df/f43434a3-e222-4b80-a4d9-2513723adb43}{Bachelor of Computer Science}
				\textsubscript{
					\href{https://www.myequals.net/sharelink/0ca660ab-7f1b-42fd-861b-e9b03df80363/0f37473b-f685-41bb-abd6-98d1e9dd7fb0}{AHEGS}
				}
			\item
				\href{https://www.myequals.net/sharelink/46bae55d-6a51-42f2-a6d3-ad1b9bab7f02/3dc307bb-09b1-43f1-b319-ee16fa643d2f}{Associates Degree in IT}
				\textsubscript{
					\href{https://www.myequals.net/sharelink/9ca4d745-4601-416f-a0fd-5c7334088e4a/49e877ab-89ee-4c36-a9d1-fd9f9911dc0a}{AEHGS}
				}
		\end{itemize}
\end{itemize}

\section{Technologies Used}

\begin{itemize}
	\item C
	\item C++
	\item Git
	\item C\#
	\item Java
	\item Bash
	\item SQL
	\item \LaTeX
	\item HTML
	\item CSS
	\item JavaScript
	\item Python
	\item Ruby
	\item Lua
\end{itemize}



\pagebreak
\section{Github}
\begin{itemize}
	\item \href{https://github.com/1wsx10/}{Personal Repository}
	\item \href{https://github.com/s3543536/}{University Projects}
\end{itemize}

\section{Projects}

\subsection{\href{https://github.com/Hazelfire/I3DAssignment2}{3D Frogger}}
\begin{itemize}
	\item C++
	\item OpenGL
	\item Cmake
	\item Boost test
\end{itemize}
This was a partner project in university where we wrote a game.
It does not use any existing game engine, it just uses OpenGL to draw
things.

\subsection{\href{https://github.com/1wsx10/VectorThrust2}{In-Game Flight Controller}}
\begin{itemize}
	\item C\#
\end{itemize}
This program runs within the game Space Engineers. It rotates
'nacelles' (a little jet engine on a motor) in order to control the
aircraft. The normal flight behavior that is built into the game does
not work with 'nacelles'.

\subsection{\href{https://github.com/1wsx10/mandelbrot}{Mandelbrot Framebuffer Renderer}}
\begin{itemize}
	\item C
	\item pthread
	\item GNU Make
	\item my framebuffer library (see below)
\end{itemize}
This project is multithreaded and multi process. It draws pretty
mandelbrot fractals. You can move the view around, and zoom.
This does not use a bignum library, so there is a limit to how far
you can zoom (although it takes a while to zoom that deep)


\subsection{\href{https://github.com/1wsx10/my_c_draw}{Direct draw to Framebuffer}}
\begin{itemize}
	\item C
	\item linux framebuffer device
	\item GNU make
\end{itemize}
This writes directly to the framebuffer device file under linux to
draw to the screen. It only works if you don't have something else
actively drawing to the screen (like an X session). I just run this
in the virtual terminal. There is currently no screenshot
functionality, so any screenshots are taken with a physical camera.



\vspace*{\fill}
\end{document}
