\documentclass{amsart}
\renewcommand{\familydefault}{\sfdefault}
\usepackage{helvet}
\title{Angel English}
%\author{Angel English}
\address{16 Cuming st Yarraville 3013}
\email{\href{mailto:angelenglish@gmail.com}{angelenglish@gmail.com}, \href{mailto:1wsx10@gmail.com}{1wsx10@gmail.com}}
%\date{August 20, 2019}

% Check if we are compiling under latex or pdflatex, and include the
% appropriate graphics package
\ifx\pdftexversion\undefined
  \usepackage[dvips]{graphicx}
\else
  \usepackage[pdftex]{graphicx}
\fi

\usepackage{hyperref}

% Define the theorem environments
\newtheorem{theorem}{Theorem}[section] % numbered like the section
\newtheorem{lemma}[theorem]{Lemma} % numbered like the theorems
\newtheorem{proposition}[theorem]{Proposition}
\newtheorem{corollary}[theorem]{Corollary}
\theoremstyle{definition} % styled differently... not italicized
\newtheorem{definition}[theorem]{Definition}
\newtheorem{conjecture}[theorem]{Conjecture}

% Let \sign show roman-style characters in math mode
\newcommand{\sign}{\mathop{\mathrm{sign}}}

%\raggedbottom % Makes the bottom margin more flexible (helpful for pictures)

\begin{document}

\begin{abstract}
 Put an abstract here when you're ready.
\end{abstract}

\maketitle

\tableofcontents

\section{Introduction}

Introduction.  We might cite bibliographic entry \cite{SJE} here.
\href{https://www.archlinux.org}{archlinux site}

\section{Section One}

Section One.  This is the section that has an in-text equation
\(e=mc^2\) and a centered equation or two:

\begin{equation}
 \int_{-\infty}^\infty e^{-t^2}dt = \sqrt{\pi}
\end{equation}

\begin{equation*} % The star suppresses the equation numbering:
 \sum_1^\infty \frac{1}{n^2} = \frac{\pi^2}{6}
\end{equation*}

And another equation using a matrix.

\begin{equation}
 \textrm{Matrix to follow\ldots }
 \Lambda =
 \begin{bmatrix}
   3        &  \nabla \\
   \partial &  b
 \end{bmatrix}
\end{equation}

Greek letters are easy to use:

\begin{equation}
 \alpha \times \delta \rightarrow \frac{\mathbb R}{\phi}
\end{equation}

\subsection{Section One Point One}

Section One Point One.  This is the section that has the figure.  It
might appear on the next page.  One can then reference it as Figure
\ref{a figure} on page \pageref{a figure}.

\begin{figure}
\begin{center}
% Leave off the file extension so that
% the package can choose from .eps and .pdf
%\includegraphics[width=3in]{circle}
% Or if you want to waste your time, make a LaTeX figure:
\begin{picture}(50,50)
\put(25,25){\circle{50}}
\end{picture}
\end{center}
\caption{This is a big circle.  \label{a figure}}
\end{figure}

\subsection{Section One Point Two}

Section One Point Two.  This is the section that has a bulleted list.
You can number the list using \texttt{enumerate} instead of
\texttt{itemize}.

\begin{itemize}
 \item Item 1
 \item Item 2
\end{itemize}

You can also look up \texttt{description} if you're interested.

\section{I forget which number}

This is the section which has a code sample:

\begin{verbatim}
     ((lambda (x)
       (list x (list (quote quote) x)))
      (quote
         (lambda (x)
           (list x (list (quote quote) x)))))
\end{verbatim}


\section{Conclusion}

Conclusion goes here.

\appendix

\section{Stuff That Doesn't Go Up There Goes Here Instead}

This item goes without notice 19 years out of 20, but occaisionally
must be surgically removed.


\begin{thebibliography}{99}

\bibitem{SJE}Sam, Jeff, Erine.  ``A \LaTeX\ Starter File.''  2003.

\bibitem{CM}Burtis, B., and James A. Morrow.  ``Inverse Problems
for Electrical Networks.''  Series on applied mathematics -- Vol.\ 13.
World Scientific, \copyright 2000.

\end{thebibliography}

\end{document}
